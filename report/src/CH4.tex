\chapter{デバイス・ドライバの作成}
4章ではデバイス・ドライバを作成する.
今回はopen, write, read, close, ioctlなどの関数を用意して読み書きや特別な操作のできるキャラクタデバイスを作成した.

\section{キャラクタデバイスの作成}
今回作成するデバイス・ドライバは読み書きとioctlによる2つの値の設定・取得・交換のできるキャラクタデバイスである.

作成したデバイス・ドライバのソースコードとヘッダをリスト\ref{lst:chdev},\ref{lst:chdevh}に示す.
\begin{longlisting}
\begin{myminted}{c}{my_chdev.c}
#include <linux/module.h>
#include <linux/kernel.h>
#include <linux/fs.h>
#include <linux/types.h>
#include <linux/cdev.h>
#include <linux/slab.h>
#include "my_chdev.h"
#define DRIVER_NAME "MyDevice"
#define BUF_SIZE 256

static const unsigned int MINOR_BASE = 0;
static const unsigned int MINOR_NUM  = 3;
static unsigned int major_num;
static struct cdev my_cdev;

MODULE_LICENSE("Dual BSD/GPL");

struct _mydevice_file_data {
    unsigned char buffer[BUF_SIZE];
    struct parameter param;
};

static long mydevice_ioctl(struct file *fp, unsigned int cmd, unsigned long arg) {
    printk("called ioctl\n");

    int tmp = 0;
    switch (cmd) {
        case MYDEVICE_SET_VALUES:
            if (copy_from_user(&(((struct _mydevice_file_data *)fp->private_data)->param), (struct parameter __user *)arg, sizeof(struct parameter))) {
                return -EFAULT;
            }
            break;
        case MYDEVICE_GET_VALUES:
            if (copy_to_user((struct parameter __user *)arg, &(((struct _mydevice_file_data *)fp->private_data)->param), sizeof(struct parameter))) {
                return -EFAULT;
            }
            break;
        case MYDEVICE_SWAP_VALUES:
            tmp = ((struct _mydevice_file_data *)fp->private_data)->param.value1;
            ((struct _mydevice_file_data *)fp->private_data)->param.value1 = ((struct _mydevice_file_data *)fp->private_data)->param.value2;
            ((struct _mydevice_file_data *)fp->private_data)->param.value2 = tmp;
            break;
        default:
            printk(KERN_WARNING "unsupported command %d\n", cmd);
            return -EFAULT;
    }

    return 0;
}

static int mydevice_open(struct inode *inode, struct file *fp) {
    printk("open my device\n");

    struct _mydevice_file_data *p = kmalloc(sizeof(struct _mydevice_file_data), GFP_KERNEL);
    if (p == NULL) {
        printk(KERN_ERR  "kmalloc\n");
        return -ENOMEM;
    }

    strlcat(p->buffer, "init", 4);
    p->param.value1 = 0;
    p->param.value2 = 0;
    fp->private_data = p;

    return 0;
}

static int mydevice_close(struct inode *inode, struct file *fp) {
    printk("close my device\n");

    if (fp->private_data) {
        kfree(fp->private_data);
    }

    return 0;
}

static ssize_t mydevice_read(struct file *fp, char __user *_buf, size_t count, loff_t *f_pos) {
    printk("read my device\n");

    if (count > BUF_SIZE) count = BUF_SIZE;

    struct _mydevice_file_data *p = fp->private_data;
    if (copy_to_user(_buf, p->buffer, count) != 0) {
        return -EFAULT;
    }
    return count;
}

static ssize_t mydevice_write(struct file *fp, const char __user *_buf, size_t count, loff_t *f_pos) {
    printk("write my device\n");

    struct _mydevice_file_data *p = fp->private_data;
    if (copy_from_user(p->buffer, _buf, count) != 0) {
        return -EFAULT;
    }
    return count;
}

struct file_operations s_mydevice_fops = {
    .open    = mydevice_open,
    .release = mydevice_close,
    .read    = mydevice_read,
    .write   = mydevice_write,
    .unlocked_ioctl = mydevice_ioctl,
};

static int mydevice_init(void) {
    printk("init my device\n");

    dev_t dev;
    if (alloc_chrdev_region(&dev, MINOR_BASE, MINOR_NUM, DRIVER_NAME) != 0) {
        printk(KERN_ERR  "alloc_chrdev_region = %d\n", major_num);
        return -1;
    }

    major_num = MAJOR(dev);
    dev = MKDEV(major_num, MINOR_BASE);

    cdev_init(&my_cdev,&s_mydevice_fops);
    my_cdev.owner = THIS_MODULE;
    int cdev_err = cdev_add(&my_cdev, dev, MINOR_NUM);
    if (cdev_err != 0) {
        printk(KERN_ERR  "cdev_add = %d\n", cdev_err);
        unregister_chrdev_region(dev, MINOR_NUM);
        return -1;
    }
    return 0;
}

static void mydevice_exit(void) {
    printk("exit my device\n");

    dev_t dev = MKDEV(major_num, MINOR_BASE);
    cdev_del(&my_cdev);
    unregister_chrdev_region(dev, MINOR_NUM);
}

module_init(mydevice_init);
module_exit(mydevice_exit);
\end{myminted}
\caption{作成したデバイス・ドライバ}
\label{lst:chdev}
\end{longlisting}


\begin{longlisting}
\begin{myminted}{c}{my_chdev.h}
#pragma once
#include <linux/ioctl.h>

struct parameter {
    int value1;
    int value2;
};

#define MYDEVICE_IOC_TYPE 'X'
#define MYDEVICE_SET_VALUES _IOW(MYDEVICE_IOC_TYPE, 1, struct parameter)
#define MYDEVICE_GET_VALUES _IOR(MYDEVICE_IOC_TYPE, 2, struct parameter)
#define MYDEVICE_SWAP_VALUES _IOR(MYDEVICE_IOC_TYPE, 3, struct parameter)
\end{myminted}
\caption{my_chdev.h}
\label{lst:chdevh}
\end{longlisting}

まずmodule_initマクロとmodule_exitマクロを使ってload時とunload時に呼ばれる関数を登録する.
次にfile_operations構造体を定義してopen, release, read, write にそれぞれ呼ばれる関数のポインタを登録する.
また,unlocked_ioctlにioctlの操作を定義する関数のポインタを登録する.
このようにすることで各動作時の振る舞いを簡単に定義することができる.

各関数の処理について説明する.
まず,mydevice_initではキャラクタデバイスの登録をしている.
alloc_chrdev_region関数を用いてメジャー番号とマイナー番号,ドライバの名前などを登録し,cdev_init関数で初期化,
cdev_addで登録を行う.
mydevice_exitでは逆にcdev_delで削除を行い,unregister_chrdev_regionで登録も取り消す.

mydevice_openではファイルポインタのprivate_dataにioctlのためのデータ領域を確保する.
グローバル変数に領域を確保しても良いが,その場合同時に複数キャラクタデバイスにアクセスがあったときにデータが1つしか
存在できないので競合が起きてしまう.
ファイルポインタの領域にkmallocを使って確保することで複数の接続でもそれぞれで専用のデータ領域を確保できる.
mydevice_closeではopenで確保した領域を確保している.

mydevice_readではファイルポインタに保存しているデータをユーザにコピーしてる.
この際copy_to_userを用いて安全にユーザ空間のデータをカーネル空間に持ち込んでいる.
mydevice_writeではreadの逆で,ユーザ空間からのデータをファイルポインタに確保している空間にコピーしている.

mydevice_ioctlではコマンドを受け取って実行する部分の処理を実装している.
MYDEVIDE_SET_VALUESでは構造体parameterを受け取ってデータ領域に値をセットする.
MYDEVICE_GET_VALUESでは保存されていた値をargに代入して返す.
MYDEVICE_SWAP_VALUESは構造体parameterに2つあるデータを交換する.

\section{キャラクタデバイスの動作確認}
上記の実装が正しく動作しているかを検証する.
キャラクタデバイスをビルドするために用いたMakefileをリスト\ref{lst:chdevmake}に示す.
\begin{longlisting}
\begin{myminted}{make}{Makefile}
obj-m := my_chdev.o

all:
	make -C /lib/modules/$(shell uname -r)/build M=$(shell pwd) modules

clean:
	make -C /lib/modules/$(shell uname -r)/build M=$(shell pwd) clean
	rm a.out

load:
	sudo insmod my_chdev.ko
	sudo mknod --mode=666 /dev/mydevice0 c `grep MyDevice /proc/devices | awk '{print \$\$1;}'` 0
	sudo mknod --mode=666 /dev/mydevice1 c `grep MyDevice /proc/devices | awk '{print \$\$1;}'` 1
	sudo mknod --mode=666 /dev/mydevice2 c `grep MyDevice /proc/devices | awk '{print \$\$1;}'` 2

unload:
	sudo rmmod my_chdev.ko
	sudo rm /dev/mydevice0
	sudo rm /dev/mydevice1
	sudo rm /dev/mydevice2


.PHONY: load remove
\end{myminted}
\caption{キャラクタデバイスをビルドするために用いたMakefile}
\label{lst:chdevmake}
\end{longlisting}

マイナー番号が機能しているか確認するために複数のデバイスを同時に作成している.
実際に動作を確認するときは以下のようにビルドして動作を確認する.
\begin{quote}
\$ make
\$ make load
\$ gcc call_mydevice.c
\$ ./a.out
\$ make unload
\end{quote}

作成したキャラクタデバイスを開いて操作するプログラムをリスト\ref{lst:callchdev}に示す.
\begin{longlisting}
\begin{myminted}{c}{call_mydevice.c}
#include <stdio.h>
#include <stdlib.h>
#include <fcntl.h>
#include <unistd.h>
#include <errno.h>
#include <sys/ioctl.h>
#include "my_chdev.h"

int main() {
    int fd;
    struct parameter values_set;
    struct parameter values_get;
    values_set.value1 = 1;
    values_set.value2 = 2;

    if ((fd = open("/dev/mydevice0", O_RDWR)) < 0) perror("open");

    if (ioctl(fd, MYDEVICE_SET_VALUES, &values_set) < 0) perror("ioctl_set");
    if (ioctl(fd, MYDEVICE_GET_VALUES, &values_get) < 0) perror("ioctl_get");
    if (ioctl(fd, MYDEVICE_SWAP_VALUES, &values_set) < 0) perror("ioctl_get");
    if (ioctl(fd, MYDEVICE_GET_VALUES, &values_get) < 0) perror("ioctl_get");

    // expected val1 = 2, val2 = 1
    printf("value1 = %d, value2 = %d\n", values_get.value1, values_get.value2);

    if (close(fd) != 0) perror("close");
    return 0;
}
\end{myminted}
\caption{作成したキャラクタデバイスを開いて操作するプログラム}
\label{lst:callchdev}
\end{longlisting}

% 動作確認
