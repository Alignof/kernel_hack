\chapter{最後に}
プログラムに作成した時間は1週間あたり授業時間の3時間$\times$2と授業時間外で平均5時間くらいはやっていたと記憶している.
つまり14週間$\times$11で大体150時間はやっていた.

レポートを作成するのにかかった時間は合わせて20時間くらいだと思う.

\section{感想}
せっかく筑波に編入したのだから難しい課題に挑戦したいということでこのカーネルハックを選択したのだが,期待通りに歯ごたえのある課題だった.
興味はあったがあまり詳しくなかったLinuxカーネルの一端にふれることができたように感じる.
今までカーネル空間でのプログラムはあまり経験がなかったが,浮動小数点が使えなかったりスタックのサイズにより厳しい制約があったり
ユーザ空間と違った世界が広がっていることが分かった.チューターさんに制約の理由を聞くとどれも納得できるような内容でレイヤの低い領域
ならではの事情があるのが面白かった.

一方で意外にもカーネル空間でもユーザ空間と同じ方法が通用する場合もあって,よく関数などを追ってみると裏で努力してカーネルでも同じような
インターフェースで使えるようにマクロが組んであったりして工夫がされているなと感じた.
特に課題6はシステムプログラムの講義と内容をリンクさせて課題を進められて感心した.

課題自体の難易度も高かったが,カーネル空間の事情に疎いのでカーネルで何ができるのかできないのかを見極めながら課題を設定したり実装する
のが思ったより大変だった.
課題6で言えばカーネル空間でサーバと通信ができるのか,できたとしてどのレイヤでなのか,HTTPSでの通信は可能なのかということが分からないまま
手探りで進めることも多く,wikiやチューターの方に助けられた.

この数ヶ月難しい課題やチューターの方とのやりとりなどで多くのカーネルにまつわる知見を得られた.
カーネルを含む低レイヤには興味があるので今後も今回の経験を活かして更に知識を深めていきたい.
