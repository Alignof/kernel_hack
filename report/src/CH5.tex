\chapter{/proc ファイル・システムの作成}
本章ではprocファイルシステムを通じてアクセス可能なモジュールを作成する.
今回はneofetch\cite{neofetch}のような複数のデバイスの情報をまとめて閲覧できるprocファイルを作成した.

\section{procファイルの実装}
今回表示する情報として,
\begin{itemize}
    \item カーネルのバージョン
    \item uptime
    \item CPUのベンダ
    \item CPUのモデル
    \item CPUの周波数
    \item キャッシュのサイズ
    \item CPUのcore id
    \item CPUのコア数
    \item メモリの総使用量
    \item 使ってないメモリ
    \item 使用中のメモリ
\end{itemize}
を選んだ.

複数のデバイスの情報をまとめて表示するにあたってcpuやメモリの情報をカーネル内の構造体から直接取得した.
実装をリスト\ref{lst:proc}に示す.
\begin{longlisting}
\begin{myminted}{c}{my\_procfs.c}
#include <linux/module.h>
#include <linux/kernel.h>
#include <linux/types.h>
#include <linux/fs.h>
#include <linux/mm.h>
#include <linux/proc_fs.h>
#include <linux/slab.h>
#include <linux/cpufreq.h>
#include <linux/utsname.h>
#include <linux/kallsyms.h>
#include <linux/kprobes.h>
#include <linux/time_namespace.h>
#include <asm/page_types.h>
#include <asm/processor.h>
#include <asm/cpufeature.h>

#include "my_procfs.h"

#define pagenum_to_KB(x) ((x) << (PAGE_SHIFT - 10))
#define DRIVER_NAME "MyDevice"
#define PROC_NAME "MyProcFs"
#define BUF_SIZE 512

static char buffer[BUF_SIZE];
static unsigned int major_num;

MODULE_LICENSE("Dual BSD/GPL");


static int mydevice_open(struct inode *inode, struct file *fp) {
    printk("open my device\n");
    int *buf_pos = kmalloc(sizeof(int), GFP_KERNEL);
    *buf_pos = BUF_SIZE;
    fp->private_data = buf_pos;
    return 0;
}

static int mydevice_close(struct inode *inode, struct file *fp) {
    printk("close my device\n");
    return 0;
}

static ssize_t mydevice_read(struct file *fp, char __user *_buf, size_t count, loff_t *f_pos) {
    const int cpu_id = 0;
    int *buf_pos = fp->private_data;
    struct cpuinfo_x86 *c = &cpu_data(cpu_id);

    if (*buf_pos == BUF_SIZE) {
        unsigned int freq = 0;
        typedef unsigned long (*kallsyms_lookup_name_t)(const char *name);
        kallsyms_lookup_name_t kallsyms_lookup_name;
        static struct kprobe kp = {
           .symbol_name = "kallsyms_lookup_name"
        };
        register_kprobe(&kp);
        kallsyms_lookup_name = (kallsyms_lookup_name_t) kp.addr;
        unregister_kprobe(&kp);
        unsigned int (*aperfmperf_get_khz)(int) = kallsyms_lookup_name("aperfmperf_get_khz");
     
        if (cpu_has(c, X86_FEATURE_TSC)) {
            if (aperfmperf_get_khz != 0) freq = aperfmperf_get_khz(cpu_id);

            if (!freq) freq = cpufreq_quick_get(cpu_id);
            if (!freq) freq = cpu_khz;
        }

        struct sysinfo mem_info;
        si_meminfo(&mem_info);

        struct timespec64 uptime;
        ktime_get_boottime_ts64(&uptime);
        timens_add_boottime(&uptime);

        *buf_pos = snprintf(
            buffer, 
            sizeof(buffer), 
            "========= system =========\n"
            "kernel version: %s %s %s\n"
            "uptime: %lu.%02lu\n"
            "========== cpu ==========\n"
            "vender id: %s\n"
            "cpu family: %u\n"
            "model: %s\n"
            "cpu MHz: %u.%03u\n"
            "cache size: %u KB\n"
            "core id: %u\n"
            "cpu cores: %u\n"
            "========= memory =========\n"
            "MemTotal: %lu\n"
            "MemFree: %lu\n"
            "MemUsed: %lu\n"
            "",
            utsname()->sysname,
            utsname()->release,
            utsname()->version,
            (unsigned long) uptime.tv_sec,
            (uptime.tv_nsec / (NSEC_PER_SEC / 100)),
            c->x86_vendor_id,
            c->x86,
            c->x86_model_id,
            freq / 1000, (freq % 1000),
            c->x86_cache_size,
            c->cpu_core_id,
            c->x86_max_cores,
            pagenum_to_KB(mem_info.totalram),
            pagenum_to_KB(mem_info.freeram),
            pagenum_to_KB(mem_info.totalram - mem_info.freeram)
        ) + 1;
        printk("buf_pos = %d", *buf_pos);
    }

    int copy_size = count;
    if (count > *buf_pos) copy_size = *buf_pos;
    copy_to_user(_buf, buffer, copy_size);

    *f_pos += copy_size;
    int i;
    for (i = copy_size; i < *buf_pos; i++) {
        buffer[i - copy_size] = buffer[i];
    }
    *buf_pos -= copy_size;
    printk( KERN_INFO "%s : buf_pos = %d\n", buffer, *buf_pos );

    printk("read my device\n");

    printk("buf_pos = %d", *buf_pos);
    printk("count = %d", count);
    printk("copy_size = %d", copy_size);

    return copy_size;
}

static ssize_t mydevice_write(struct file *fp, const char __user *_buf, size_t count, loff_t *f_pos) {
    printk("write my device\n");

    return count;
}

static struct proc_ops s_mydevice_fops = {
    .proc_open    = mydevice_open,
    .proc_release = mydevice_close,
    .proc_read    = mydevice_read,
    .proc_write   = mydevice_write,
};

static int mydevice_init(void) {
    printk("init my device\n");

    struct proc_dir_entry *entry;
    entry = proc_create(PROC_NAME, S_IRUGO | S_IWUGO, NULL, &s_mydevice_fops);
    if (entry == NULL) {
        printk(KERN_ERR, "proc_create\n");
        return -ENOMEM;
    }

    return 0;
}

static void mydevice_exit(void) {
    printk("exit my device\n");

    remove_proc_entry(PROC_NAME, NULL);
}

module_init(mydevice_init);
module_exit(mydevice_exit);
\end{myminted}
\caption{procファイルの実装}
\label{lst:proc}
\end{longlisting}

基本的にmodule\_initやmodule\_exitでモジュールの初期化と終了時の関数を設定すること,構造体に関数ポインタを渡してopen, release, read, write
時の動作を定義することは4章と同じである.
5章では読み込みのみ対応すれば良いとのことなのでwrite関数については実装を省略している.
以下,それぞれのデバイスのデータの取得について説明する.

\subsection{kernel version}
カーネルのバージョンはutsnameという関数からnew\_utsnameという構造体を経由して簡単に参照することができる.\cite{utsname}
定義自体もinclude可能なヘッダに存在するので呼び出すだけで簡単に参照することができる.
バージョン名はsysnameとreleaseとversionに分かれている.

\subsection{uptime}
uptimeはtimespec64という構造体の中に格納されるものであると分かった.\cite{timespec64}
まずktime\_get\_boottime\_ts64\footnote{マクロでns\_to\_timespec64に置き換わっている}でuptimeに起動時間を格納する.
更にtimens\_add\_boottimeでbootにかかった時間を足している.\cite{time_add}

\subsection{cpu}
cpuの情報は構造体cpuinfo_x86を介してcpu_dataから取得することができる.
ただし,cpuinfo_x86の定義を見れば分かるように周波数の情報だけは格納されてない.\cite{cpuinfo}
そのため周波数は別の方法で取得する必要がある.

しばらく調べるとaperfmperf_get_khzという関数を使ってcpuのidを引数に周波数を取得できる\cite{aperf}ことが分かったが,
ここで1つ問題が生じる.
周波数以外の情報を取得するための関数は全て外部から参照可能なもので適切にヘッダファイルをincludeすれば参照可能だった.
%例えば

しかし,このaperfmperf_get_khzはシンボルがエクスポートされておらずどこからも参照ができない.
対処方法として,
\begin{enumerate}
    \item カーネルのソースコードからコピーする.
    \item この関数を使わずに参照できる関数を使って同様の処理を実現する.
    \item includeする以外の方法でシンボルを読み込む.
\end{enumerate}

という方法が考えられる.
1.は単純な方法だが,ファイルを超えて関数や変数をコピーするたびに参照するべきものが増えてしまい(結果的に収束するにせよ)
爆発的な量のソースコードのコピーが発生するため避けた.
2.の可能性も考慮したが,手元で試す限りやはりaperfmperf_get_khzが周波数を取得するためこれも難しいと判断した.
そこで今回は3の方法を使ってシンボルがエクスポートされてない関数を参照して使うことにした.

シンボルがエクスポートされていない関数を参照する方法としてkallsyms_lookup_nameという関数を使う選択肢がある.
しかし,肝心のkallsyms_lookup_nameもシンボルがエクスポートされていない.
古いLinuxカーネルでは有効になっていたが,エクスポートされていない関数のアドレスも返してしまい危険だということで
5.7以降でエクスポートされなくなっていた.\cite{kallsyms}

そこで今回はkprobeを介してまずkallsyms_lookup_nameの関数ポインタを取得し,
kallsyms_lookup_nameを使ってaperfmperf_get_khzの関数ポインタを取得する方法を採用した.
まず構造体kprobeにシンボル名を代入し,register_kprobeで登録する.
すると構造体kprobeのaddrに関数のアドレスが入るのでそれを取得することでkallsyms_lookup_nameの参照を獲得し,
参照可能になったkallsyms_lookup_nameを使ってaperfmperf_get_khzを参照することができた.
\footnote{簡単そうに言っているが方法があるのかわからない状態で手探りでこれをやるのは結構大変だった.
が,案外上手く行ってしまいカーネル空間の力の強さを感じた.}

この方法によって無事周波数についても値を取得することができた.

\subsection{memory}
メモリに関する情報は構造体sysinfoを通してsi\_meminfoから取得可能である.\cite{meminfo}
構造体sysinfoにはメモリに関する情報がきれいに全部入っているのでこのまま使用可能である.

\subsection{表示}
ファイルの表示は先にsnprintfを使って整形した状態でバッファに書き込み
それをcopy_to_userを使ってユーザ空間の配列にコピーする.
一度で読みきれない可能性もあるので読んだところまでを記録しておき次読むときに備える.
全て読まれた(残りのバッファが0になった)とき新しくデータを取得する.

\section{動作の確認}
上記の実装が正しく動作しているかを検証する.
procファイルシステムを通じてアクセス可能なモジュールをビルドするために4章と同様にMakefileを用いた.
使用したMakefileをリスト\ref{lst:procmake}に示す.
\begin{longlisting}
\begin{myminted}{make}{Makefile}
obj-m := my_procfs.o
EXTRA_CFLAGS := -I/home/tsukubataro/kernel_hack/linux-5.17.1/arch

all:
	make -C /lib/modules/$(shell uname -r)/build M=$(shell pwd) modules

clean:
	make -C /lib/modules/$(shell uname -r)/build M=$(shell pwd) clean

load:
	sudo insmod my_procfs.ko

unload:
	sudo rmmod my_procfs.ko

.PHONY: load remove
\end{myminted}
\caption{5章で実装したモジュールをビルドするために用いたMakefile}
\label{lst:procmake}
\end{longlisting}

実際にファイルを読んでみる.
ファイルを読むにはmakeを使って以下のようにする.
\begin{quote}
\$ make
\$ make load
\$ cat /proc/MyProcFs
\$ make unload
\end{quote}

catの出力をリスト\ref{lst:cat}に示す.
\begin{longlisting}
\begin{myminted}{}{}
========= system =========
kernel version: Linux 5.17.1 #AddMySyscall SMP PREEMPT Sat Apr 30 02:00:12 JST 2022
uptime: 2587.05
========== cpu ==========
vender id: GenuineIntel
cpu family: 6
model: Intel(R) Core(TM) i7-6700K CPU @ 4.00GHz
cpu MHz: 4007.998
cache size: 8192 KB
core id: 0
cpu cores: 1
========= memory =========
MemTotal: 3986948
MemFree: 2003176
MemUsed: 1983772
\end{myminted}
\caption{cat /proc/MyProcFsの出力}
\label{lst:cat}
\end{longlisting}

正しくデータが表示されてそうである.
procファイルの表示と突き合わせてみる.
\begin{quote}
\$ cat /proc/uptime 
2760.60 5322.63
\$ uname -a
Linux ubuntu 5.17.1 #AddMySyscall SMP PREEMPT Sat Apr 30 02:00:12 JST 2022 x86_64 x86_64 x86_64 GNU/Linux
\$ cat /proc/meminfo | head -n 5
MemTotal:        3986948 kB
MemFree:         2003200 kB
MemAvailable:    2756972 kB
Buffers:           58176 kB
Cached:           889228 kB
\$ cat /proc/cpuinfo | head -n 15
processor	: 0
vendor_id	: GenuineIntel
cpu family	: 6
model		: 94
model name	: Intel(R) Core(TM) i7-6700K CPU @ 4.00GHz
stepping	: 3
microcode	: 0xf0
cpu MHz		: 4007.998
cache size	: 8192 KB
physical id	: 0
siblings	: 1
core id		: 0
cpu cores	: 1
apicid		: 0
initial apicid	: 0
\end{quote}

正しく表示されていることが確認できた.


