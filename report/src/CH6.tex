\chapter{定期的な仕事}
本章ではkthreadを使ってカーネル内で定期的に動作するプログラムを作成する.

\section{kthreadによる定期的な仕事}
今回作成するのはカーネル空間から定期的にサーバにアクセスし,円相場を取得してカーネルログに流すプログラムである.
作成したプログラムをリスト\ref{lst:rate}に示す.
\begin{longlisting}
\begin{myminted}{basemake}{kthread\_test.c}
#include <linux/module.h>
#include <linux/delay.h>
#include <linux/kthread.h>
#include <linux/string.h>
#include <linux/file.h>
#include <linux/in.h>
#include <linux/inet.h>
#include <linux/socket.h>
#include <linux/tls.h>
#include <net/sock.h>
#include <linux/init.h>
#define BUFFER_SIZE 1024
#define API_KEY "xxxxxxxxxxxxxxxxxxxxxxxxxxxxxxxxxxxxxxxxxxxx"

MODULE_LICENSE("Dual BSD/GPL");

struct task_struct *k;

static int kthread_main(void) {
    struct socket *sock;
    struct sockaddr_in s_addr;
    unsigned short port_num = 80;
    int ret = 0;
    char *send_buf = NULL;
    char *recv_buf = NULL;
    struct kvec send_vec, recv_vec;
    struct msghdr send_msg, recv_msg;

    /* kmalloc a send buffer*/
    send_buf = kmalloc(BUFFER_SIZE, GFP_KERNEL);
    if (send_buf == NULL) {
        printk("client: send_buf kmalloc error!\n");
        return ENOMEM;
    }

    /* kmalloc a receive buffer*/
    recv_buf = kmalloc(BUFFER_SIZE, GFP_KERNEL);
    if(recv_buf == NULL){
        printk("client: recv_buf kmalloc error!\n");
        return ENOMEM;
    }
    memset(&s_addr, 0, sizeof(s_addr));
    s_addr.sin_family = AF_INET;
    s_addr.sin_port = htons(port_num);
    s_addr.sin_addr.s_addr = in_aton("52.73.174.144");
    sock = (struct socket *)kmalloc(sizeof(struct socket), GFP_KERNEL);

    ret = sock_create_kern(&init_net, AF_INET, SOCK_STREAM, IPPROTO_TCP, &sock);
    if (ret < 0) {
        printk("client:socket create error!\n");
        return ret;
    } 

    ret = sock->ops->connect(sock, (struct sockaddr *)&s_addr, sizeof(s_addr), 0);
    if (ret != 0) {
        printk("client: connect error!\n");
        return ret;
    }
    memset(send_buf, 0, BUFFER_SIZE);
    snprintf(
        send_buf,
        BUFFER_SIZE,
        "GET /api/latest.json?app_id="
        API_KEY
        "&symbols=JPY HTTP/1.1\r\n"
        "Host: 52.73.174.144\r\n"
        "\r\n"
    );
    memset(&send_msg, 0, sizeof(send_msg));
    memset(&send_vec, 0, sizeof(send_vec));

    send_vec.iov_base = send_buf;
    send_vec.iov_len = BUFFER_SIZE;

    ret = kernel_sendmsg(sock, &send_msg, &send_vec, 1, BUFFER_SIZE);
    if (ret < 0) {
        printk("client: kernel_sendmsg error!\n");
        return ret;
    }

    memset(recv_buf, 0, BUFFER_SIZE);
    memset(&recv_vec, 0, sizeof(recv_vec));
    memset(&recv_msg, 0, sizeof(recv_msg));
    recv_vec.iov_base = recv_buf;
    recv_vec.iov_len = BUFFER_SIZE;

    ret = kernel_recvmsg(sock, &recv_msg, &recv_vec, 1, BUFFER_SIZE, 0);
    char *jpy = NULL;
    if (jpy = strstr(recv_buf, "\"JPY\": ")) {
        jpy += sizeof("\"JPY\": ")-1;
    }
    printk("1USD = %.*sJPY\n", 8, jpy);

    kernel_sock_shutdown(sock, SHUT_RDWR);
    sock_release(sock);

    return 0;
}

static int kthread_function(void* arg) {
    printk(KERN_INFO "[%s] start kthread\n", k->comm);

    while (!kthread_should_stop()) {
        kthread_main();
        ssleep(3600);
    }

    printk(KERN_INFO "[%s] stop kthread\n", k->comm);

    return 0;
}

static int __init testmod_init(void) {
    printk(KERN_INFO "driver loaded\n");

    k = kthread_run(kthread_function, NULL, "testmod kthread");

    return 0;
}

static void __exit testmod_exit(void) {
    kthread_stop(k);
    printk(KERN_INFO "driver unloaded\n");
}

module_init(testmod_init);
module_exit(testmod_exit);
\end{myminted}
\caption{円相場を取得してカーネルログに流すプログラム}
\label{lst:rate}
\end{longlisting}

このプログラムはkthreadを用いて1時間(3600秒)ごとにサーバにアクセスしてドル円のデータを取得するプログラムである.
出力はprintkを用いてカーネルログに流す.
定期的な実行とデータの取得の観点から以下プログラムについて説明する.

\subsection{定期実行}
カーネル内での定期的な実行にはkthreadを用いた.
まず初期化の際にkthread\_runを呼び出して第1引数に渡したkthread\_functionを呼び出しスレッドにする.
この際返り値として構造体task\_structが帰ってくるのでグローバル変数として保存しておく.
これはモジュールを終了するときにスレッドを止めるのに用いる.

kthread\_functionではkthread\_should\_stopがtrueになるまでの間実行を繰り返す.
実際の処理はkthread\_mainにかかれておりkthread\_mainを実行するのと1時間待機するのを繰り返すことで定期的な仕事を実現している.

\subsection{データの取得}
データの取得はkthread\_mainで行っている.
為替をAPIで取得できるopen exchange rates\footnote{\url{https://openexchangerates.org/}}のサービスに登録し,そこのAPIを叩いた.
最初は別のサービスのAPIの使用を検討していたが,HTTPSでしかリクエストを受け付けないサービスは実装が難しいと判断してHTTPでもリクエスト
を送信できるこのサービスに決めた.今後機会があればSSLを用いた通信にも挑戦したい.

通信はソケットを作成しTCPでHTTPのリクエストを送信することでデータを取得できた.
通信を行っている関数kthread\_mainの詳しい仕様をmanの形式に倣って以下に示す.
\begin{longlisting}
\begin{myminted}{text}[]
\$ man 2 kthread_main
NAME
       kthread_main - get a dollar to yen exchange rate

SYNOPSIS
       static int kthread_main(void)

DESCRIPTION
       カーネル内でHTTPを用いてサーバにアクセスし円相場(1USDあたりのJPY)を取得する.

RETURN VALUE
       成功した場合,0が返される.
       バッファの確保に失敗した場合,ソケットの作成,接続,メッセージの送信などに失敗されたとき内容に応じて下記に示すエラーが返される.

ERRORS
       EINVAL ソケットのタイプが不正な場合に返される
       ENFILE ソケットをこれ以上開けない場合に返される.
       ENOMEM バッファの確保に失敗した場合に返される.
\end{myminted}
\caption{kthread\_mainの仕様}
\label{lst:syscallspec}
\end{longlisting}

処理は単純でシステムプログラムの課題でやったようなユーザランドのプログラムとあまり変わらなかった.
まずport番号を80番に設定しkmallocでバッファをそれぞれ確保する.
その後構造体sockaddr\_inに値を設定してソケットをそれをsock\_create\_kernに渡すことでソケットを作成する.
sockaddr\_inに渡すIPアドレスはurlから事前にdigコマンドで取得したものである.

ソケットを取得したら接続しHTTPのリクエストを送信する.
今回であれば/api/latest.jsonにAPI KEYであるapp\_idと対象となる通貨のコードであるsymbolsをパラメタとして渡せばいい\cite{api}ので
GETを使ってリクエストを送信する.
リスト\ref{lst:rate}のAPI KEYはダミーのものを記載しているが,実際にはサービスに登録した際に取得した値が入っている.

kernel\_sendmsgとkernel\_recvmsgで送受信を行い,サーバから送られた情報がjson形式でrecv\_bufに格納される.
今回はjsonパーサを利用したり手書きせずにデータがある位置をstrstrで探してそこから決まった文字数を取得することでデータを獲得した.
円相場が1ドル1000円台にならない限り小数点4桁までをカーネルログに表示する.


\section{動作の確認}
実際に動かして確かめる.
使用したMakefileをリスト\ref{lst:kthreadmake}に示す.
\begin{longlisting}
\begin{myminted}{basemake}{Makefile}
obj-m := kthread_test.o

all:
	make INCDIR=/usr/include -C /lib/modules/$(shell uname -r)/build M=$(shell pwd) modules

clean:
	make -C /lib/modules/$(shell uname -r)/build M=$(shell pwd) clean

load:
	sudo insmod kthread_test.ko

unload:
	sudo rmmod kthread_test.ko

.PHONY: load remove
\end{myminted}
\caption{6章で実装したモジュールをビルドするために用いたMakefile}
\label{lst:kthreadmake}
\end{longlisting}

実際に実行する際にはmakeを使って以下のようにする.
\begin{quote}
\$ make \\
\$ make load \\
\$ dmesg -w \\
\$ make unload
\end{quote}

検証する際には間隔を1時間にして実行した.
dmesgの出力をリスト\ref{lst:dmesg}に示す.
\begin{longlisting}
\begin{myminted}{text}[]
[ 7221.627893] driver loaded
[ 7221.627972] [testmod kthread] start kthread
[ 7222.049732] 1USD = 133.2538JPY
[10917.538824] 1USD = 133.2505JPY
[14587.487342] 1USD = 133.2282JPY
[18257.578345] 1USD = 133.3325JPY
[21927.805566] 1USD = 133.2849JPY
[25522.518082] [testmod kthread] stop kthread
[25522.518149] driver unloaded
\end{myminted}
\caption{dmesgの出力}
\label{lst:dmesg}
\end{longlisting}

本来カーネルログには様々なデータが流れるが,関連するもののみを抜粋して掲載した.
1時間毎にドル円の為替の動きが表示されていることが分かる.
正しく動作していることが分かった.
